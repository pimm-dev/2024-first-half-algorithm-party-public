\begin{problem}{수열 회전과 쿼리}{standard input}{standard output}{2 seconds}{1024 megabytes}

길이가 $N$인 정수 수열 $A_1,A_2,\dots,A_N$이 주어진다. 이때 다음 쿼리를 수행하는 프로그램을 작성하시오.

--- $1$ $k$ : 수열을 오른쪽으로 $k$칸 회전시킨다. 즉, 수열은 $A_{N-k+1},A_{N-k+2},\dots,A_N,A_1,A_2,\dots,A_{N-k}$로 변한다.

--- $2$ $k$ : 수열을 왼쪽으로 $k$칸 회전시킨다. 즉, 수열은 $A_{k+1},A_{k+2},\dots,A_N,A_1,A_2,\dots,A_{k}$로 변한다.

--- $3$ $a$ $b$ : 수열의 $a$번째 수부터 $b$번째 수의 합을 출력한다.

**수열 $[1,2,3,4]$을 오른쪽으로 $1$칸 회전시킨다면 $[4,1,2,3]$, 왼쪽으로 $1$칸 회전시킨다면 $[2,3,4,1]$이다.


제한

$2 \le N \le 200\,000$

$1 \le Q \le 200\,000$

$1 \le A_i \le 10^9$

$1 \le k \le N$

$1 \le a \le b \le N$

입력으로 주어지는 모든 수는 정수이다.

$3$번 쿼리는 한 번 이상 주어진다.

\InputFile
첫 번째 줄에 수열의 길이 $N$과 쿼리의 수 $Q$가 공백으로 구분되어 주어진다.
두 번째 줄에 $N$개의 정수 $A_1,A_2,\dots,A_N$이 공백으로 구분되어 주어진다.
세 번째 줄부터 $Q$개의 줄에 쿼리가 한 줄에 하나씩 주어진다.

\OutputFile
$3$번 쿼리에 대한 결괏값을 한 줄에 하나씩 입력으로 주어진 순서대로 출력한다.

\Example

\begin{example}
\exmpfile{example.01}{example.01.a}%
\end{example}

\Note
$1$. 수열 $[3,2,1,5,10,6,2]$에서 $1$번째 수부터 $3$번째 수의 합은 $6$이다.

$2$. 수열 $[3,2,1,5,10,6,2]$에서 왼쪽으로 $1$칸 회전시켰으니 수열이 $[2,1,5,10,6,2,3]$으로 변한다.

$3$. 수열 $[2,1,5,10,6,2,3]$에서 $1$번째 수부터 $3$번째 수의 합은 $8$이다.

$4$. 수열 $[2,1,5,10,6,2,3]$에서 오른쪽으로 $3$칸 회전시켰으니 수열이 $[6,2,3,2,1,5,10]$으로 변한다.

$5$. 수열 $[6,2,3,2,1,5,10]$에서 $2$번째 수부터 $6$번째 수의 합은 $13$이다.

\end{problem}

