\begin{problem}{시계탑}{standard input}{standard output}{1 second}{256 megabytes}

근성은 공대 7호관 옆 카페에서 자바칩 프라푸치노를 주문하려다 문득 생각이 나 카페 직원에게 물어보았다. 'c++칩 프라푸치노는 없나요?' 
근성의 황당한 질문을 들은 카페 직원은 화가 나서 근성을 바로 앞에 있는 시계탑 분침에 묶어버렸다. 

그러자 시계탑의 분침이 근성의 무게로 인해 이상하게 돌아가기 시작했다. 분침은 정확히 다음과 같이 순서대로 동작한다. 

1. 매 시간 정각에 분침은 12시 방향을 가리킨다. 

2. $15$분 간 분침은 시계 방향으로 정상적인 분침의 속도의 $2$배로 움직이며, 매 시간 15분이 되는 시점에 분침은 6시 방향을 가리킨다. 

3. $45$분 간 분침은 시계 방향으로 정상적인 분침의 속도의 $\frac{2}{3}$배로 움직이며, 다시 정각이 되는 시점에 분침은 12시 방향을 가리킨다.

정상적인 시계탑의 분침은 매시 정각에 정확히 12시 방향을 가리키며, $1$시간에 한 바퀴를 시계 방향으로 등속도로 움직인다. 

바뀐 시계탑의 분침도 매 시간 한 바퀴를 돌아가기 때문에 시침을 보는 것은 문제가 없었으나, 사람들은 지금이 몇 분인지 헷갈려하기 시작했다. 

이때 바뀐 분침이 가리키는 분을 나타내는 정수가 주어졌을 때, 올바른 시각을 출력하는 프로그램을 작성하시오. 

\InputFile
바뀐 분침이 가리키는 분을 나타내는 정수 $M$($0\leq M\leq 59$)이 주어진다. 

\OutputFile
올바른 시각의 분을 나타내는 실수를 정확하게 소수점 첫째 자리까지 반올림해서 출력한다. 소수점 둘째 자리가 $4$ 이하일 경우 버림, $5$ 이상일 경우 올림한다. 

\Examples

\begin{example}
\exmpfile{example.01}{example.01.a}%
\exmpfile{example.02}{example.02.a}%
\end{example}

\end{problem}

