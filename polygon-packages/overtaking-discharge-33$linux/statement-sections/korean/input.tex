첫째 줄에 종현의 전역 예정일 년 $Y_1$ $(1\,990 ≤ Y_1 ≤ 3\,000)$, 월 $M_1$, 일 $D_1$이 `YYYY MM DD` 형식으로 주어진다. 올바르지 않은 날짜는 주어지지 않는다.  

둘째 줄에 영도의 전역 예정일 년 $Y_2$ $(1\,990 ≤ Y_2 ≤ 3\,000)$, 월 $M_2$, 일 $D_2$이 `YYYY MM DD` 형식으로 주어진다. 올바르지 않은 날짜는 주어지지 않는다.

셋째 줄에 영도가 계획을 실행하는 데 사용할 수 있는 여유 $T$ $(0 \le T \le 10\,000)$와 계획을 위해 할 수 있는 행동의 개수 $N$ $(0 \le N \le 100)$이 공백으로 구분되어 주어진다.

넷째 줄부터 $N$개 줄에 걸쳐 계획을 위해 할 수 있는 행동이 아래와 같이 주어진다. 주어지는 각 행동은 최대 한 번 할 수 있다.

 - \t{1 C D} --- 영도가 $C$ $(0 \le C \le 10\,000)$만큼의 여유를 투자해서 휴가 $D$ $(1 \le D \le 30)$일을 받아낸다. 영도가 획득한 휴가는 이후 자신의 조기 전역에 사용한다.

 - \t{2 C D} --- 영도가 $C$ $(0 \le C \le 10\,000)$만큼의 여유를 투자해서 거대한 음모를 세운다. 종현은 거대한 음모에 휘말려 징계를 받고 $D$ $(1 \le D \le 15)$일 동안 군기교육대에 다녀온다.

 - \t{3 C M} --- 영도가 $C$ $(0 \le C \le 10\,000)$만큼의 여유를 투자해서 종현의 $M$개월$(6 \le M \le 48)$ 임기제 부사관 지원서를 대신 작성한다.

두 사람의 전역일은 입력으로 주어진 영도의 행동 외에 다른 요인으로 변화하지 않는다. 주어지는 수는 모두 정수이다.
