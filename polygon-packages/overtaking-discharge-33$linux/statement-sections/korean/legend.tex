영도는 국방의 의무를 다하기 위해 대한민국 육군에 입대했다. 고된 훈련병 생활과 후반기 교육을 마치고 자대 배치를 받고 나니, 자대에는 먼저 입대한 친구 종현이 있었다.

영도는 아무리 생각해도 종현이 먼저 전역하는 것이 마음에 들지 않았다. 그래서 아래의 제도와 규정을 활용해서 종현보다 먼저 전역할 계획을 세웠다.

- 조기 전역: 모아놓은 휴가를 말년에 전역일을 앞당기는 데 사용할 수 있다.

- 군기교육대: 징계를 받아 군기교육대에 가게 되면, 군기교육대에 있던 기간만큼 전역이 늦어진다.

- 임기제 부사관: 임기제 부사관에 지원하여 $6$개월에서 $48$개월까지 전문하사로 부대에 남아 국가 안보에 조금 더 이바지할 수 있다. 임기제 부사관 지원서에서의 $1$개월은 $30$일로 계산된다.

영도는 이 제도와 규정들을 적절하게 활용하여 영도의 전역일을 앞당기거나 종현의 전역일을 뒤로 미루려고 한다. 계획을 남에게 들켰다간 오히려 본인의 전역일이 늦춰질 수 있어서, 계획을 몰래 실행할 수 있는 여유 내에서 계획을 계속하려고 한다.
